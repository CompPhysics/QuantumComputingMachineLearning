\documentclass[11pt]{article}

    \usepackage[breakable]{tcolorbox}
    \usepackage{parskip} % Stop auto-indenting (to mimic markdown behaviour)
    

    % Basic figure setup, for now with no caption control since it's done
    % automatically by Pandoc (which extracts ![](path) syntax from Markdown).
    \usepackage{graphicx}
    % Maintain compatibility with old templates. Remove in nbconvert 6.0
    \let\Oldincludegraphics\includegraphics
    % Ensure that by default, figures have no caption (until we provide a
    % proper Figure object with a Caption API and a way to capture that
    % in the conversion process - todo).
    \usepackage{caption}
    \DeclareCaptionFormat{nocaption}{}
    \captionsetup{format=nocaption,aboveskip=0pt,belowskip=0pt}

    \usepackage{float}
    \floatplacement{figure}{H} % forces figures to be placed at the correct location
    \usepackage{xcolor} % Allow colors to be defined
    \usepackage{enumerate} % Needed for markdown enumerations to work
    \usepackage{geometry} % Used to adjust the document margins
    \usepackage{amsmath} % Equations
    \usepackage{amssymb} % Equations
    \usepackage{textcomp} % defines textquotesingle
    % Hack from http://tex.stackexchange.com/a/47451/13684:
    \AtBeginDocument{%
        \def\PYZsq{\textquotesingle}% Upright quotes in Pygmentized code
    }
    \usepackage{upquote} % Upright quotes for verbatim code
    \usepackage{eurosym} % defines \euro

    \usepackage{iftex}
    \ifPDFTeX
        \usepackage[T1]{fontenc}
        \IfFileExists{alphabeta.sty}{
              \usepackage{alphabeta}
          }{
              \usepackage[mathletters]{ucs}
              \usepackage[utf8x]{inputenc}
          }
    \else
        \usepackage{fontspec}
        \usepackage{unicode-math}
    \fi

    \usepackage{fancyvrb} % verbatim replacement that allows latex
    \usepackage{grffile} % extends the file name processing of package graphics
                         % to support a larger range
    \makeatletter % fix for old versions of grffile with XeLaTeX
    \@ifpackagelater{grffile}{2019/11/01}
    {
      % Do nothing on new versions
    }
    {
      \def\Gread@@xetex#1{%
        \IfFileExists{"\Gin@base".bb}%
        {\Gread@eps{\Gin@base.bb}}%
        {\Gread@@xetex@aux#1}%
      }
    }
    \makeatother
    \usepackage[Export]{adjustbox} % Used to constrain images to a maximum size
    \adjustboxset{max size={0.9\linewidth}{0.9\paperheight}}

    % The hyperref package gives us a pdf with properly built
    % internal navigation ('pdf bookmarks' for the table of contents,
    % internal cross-reference links, web links for URLs, etc.)
    \usepackage{hyperref}
    % The default LaTeX title has an obnoxious amount of whitespace. By default,
    % titling removes some of it. It also provides customization options.
    \usepackage{titling}
    \usepackage{longtable} % longtable support required by pandoc >1.10
    \usepackage{booktabs}  % table support for pandoc > 1.12.2
    \usepackage{array}     % table support for pandoc >= 2.11.3
    \usepackage{calc}      % table minipage width calculation for pandoc >= 2.11.1
    \usepackage[inline]{enumitem} % IRkernel/repr support (it uses the enumerate* environment)
    \usepackage[normalem]{ulem} % ulem is needed to support strikethroughs (\sout)
                                % normalem makes italics be italics, not underlines
    \usepackage{mathrsfs}
    

    
    % Colors for the hyperref package
    \definecolor{urlcolor}{rgb}{0,.145,.698}
    \definecolor{linkcolor}{rgb}{.71,0.21,0.01}
    \definecolor{citecolor}{rgb}{.12,.54,.11}

    % ANSI colors
    \definecolor{ansi-black}{HTML}{3E424D}
    \definecolor{ansi-black-intense}{HTML}{282C36}
    \definecolor{ansi-red}{HTML}{E75C58}
    \definecolor{ansi-red-intense}{HTML}{B22B31}
    \definecolor{ansi-green}{HTML}{00A250}
    \definecolor{ansi-green-intense}{HTML}{007427}
    \definecolor{ansi-yellow}{HTML}{DDB62B}
    \definecolor{ansi-yellow-intense}{HTML}{B27D12}
    \definecolor{ansi-blue}{HTML}{208FFB}
    \definecolor{ansi-blue-intense}{HTML}{0065CA}
    \definecolor{ansi-magenta}{HTML}{D160C4}
    \definecolor{ansi-magenta-intense}{HTML}{A03196}
    \definecolor{ansi-cyan}{HTML}{60C6C8}
    \definecolor{ansi-cyan-intense}{HTML}{258F8F}
    \definecolor{ansi-white}{HTML}{C5C1B4}
    \definecolor{ansi-white-intense}{HTML}{A1A6B2}
    \definecolor{ansi-default-inverse-fg}{HTML}{FFFFFF}
    \definecolor{ansi-default-inverse-bg}{HTML}{000000}

    % common color for the border for error outputs.
    \definecolor{outerrorbackground}{HTML}{FFDFDF}

    % commands and environments needed by pandoc snippets
    % extracted from the output of `pandoc -s`
    \providecommand{\tightlist}{%
      \setlength{\itemsep}{0pt}\setlength{\parskip}{0pt}}
    \DefineVerbatimEnvironment{Highlighting}{Verbatim}{commandchars=\\\{\}}
    % Add ',fontsize=\small' for more characters per line
    \newenvironment{Shaded}{}{}
    \newcommand{\KeywordTok}[1]{\textcolor[rgb]{0.00,0.44,0.13}{\textbf{{#1}}}}
    \newcommand{\DataTypeTok}[1]{\textcolor[rgb]{0.56,0.13,0.00}{{#1}}}
    \newcommand{\DecValTok}[1]{\textcolor[rgb]{0.25,0.63,0.44}{{#1}}}
    \newcommand{\BaseNTok}[1]{\textcolor[rgb]{0.25,0.63,0.44}{{#1}}}
    \newcommand{\FloatTok}[1]{\textcolor[rgb]{0.25,0.63,0.44}{{#1}}}
    \newcommand{\CharTok}[1]{\textcolor[rgb]{0.25,0.44,0.63}{{#1}}}
    \newcommand{\StringTok}[1]{\textcolor[rgb]{0.25,0.44,0.63}{{#1}}}
    \newcommand{\CommentTok}[1]{\textcolor[rgb]{0.38,0.63,0.69}{\textit{{#1}}}}
    \newcommand{\OtherTok}[1]{\textcolor[rgb]{0.00,0.44,0.13}{{#1}}}
    \newcommand{\AlertTok}[1]{\textcolor[rgb]{1.00,0.00,0.00}{\textbf{{#1}}}}
    \newcommand{\FunctionTok}[1]{\textcolor[rgb]{0.02,0.16,0.49}{{#1}}}
    \newcommand{\RegionMarkerTok}[1]{{#1}}
    \newcommand{\ErrorTok}[1]{\textcolor[rgb]{1.00,0.00,0.00}{\textbf{{#1}}}}
    \newcommand{\NormalTok}[1]{{#1}}

    % Additional commands for more recent versions of Pandoc
    \newcommand{\ConstantTok}[1]{\textcolor[rgb]{0.53,0.00,0.00}{{#1}}}
    \newcommand{\SpecialCharTok}[1]{\textcolor[rgb]{0.25,0.44,0.63}{{#1}}}
    \newcommand{\VerbatimStringTok}[1]{\textcolor[rgb]{0.25,0.44,0.63}{{#1}}}
    \newcommand{\SpecialStringTok}[1]{\textcolor[rgb]{0.73,0.40,0.53}{{#1}}}
    \newcommand{\ImportTok}[1]{{#1}}
    \newcommand{\DocumentationTok}[1]{\textcolor[rgb]{0.73,0.13,0.13}{\textit{{#1}}}}
    \newcommand{\AnnotationTok}[1]{\textcolor[rgb]{0.38,0.63,0.69}{\textbf{\textit{{#1}}}}}
    \newcommand{\CommentVarTok}[1]{\textcolor[rgb]{0.38,0.63,0.69}{\textbf{\textit{{#1}}}}}
    \newcommand{\VariableTok}[1]{\textcolor[rgb]{0.10,0.09,0.49}{{#1}}}
    \newcommand{\ControlFlowTok}[1]{\textcolor[rgb]{0.00,0.44,0.13}{\textbf{{#1}}}}
    \newcommand{\OperatorTok}[1]{\textcolor[rgb]{0.40,0.40,0.40}{{#1}}}
    \newcommand{\BuiltInTok}[1]{{#1}}
    \newcommand{\ExtensionTok}[1]{{#1}}
    \newcommand{\PreprocessorTok}[1]{\textcolor[rgb]{0.74,0.48,0.00}{{#1}}}
    \newcommand{\AttributeTok}[1]{\textcolor[rgb]{0.49,0.56,0.16}{{#1}}}
    \newcommand{\InformationTok}[1]{\textcolor[rgb]{0.38,0.63,0.69}{\textbf{\textit{{#1}}}}}
    \newcommand{\WarningTok}[1]{\textcolor[rgb]{0.38,0.63,0.69}{\textbf{\textit{{#1}}}}}


    % Define a nice break command that doesn't care if a line doesn't already
    % exist.
    \def\br{\hspace*{\fill} \\* }
    % Math Jax compatibility definitions
    \def\gt{>}
    \def\lt{<}
    \let\Oldtex\TeX
    \let\Oldlatex\LaTeX
    \renewcommand{\TeX}{\textrm{\Oldtex}}
    \renewcommand{\LaTeX}{\textrm{\Oldlatex}}
    % Document parameters
    % Document title
    \title{QuantumFourierTransform}
    
    
    
    
    
% Pygments definitions
\makeatletter
\def\PY@reset{\let\PY@it=\relax \let\PY@bf=\relax%
    \let\PY@ul=\relax \let\PY@tc=\relax%
    \let\PY@bc=\relax \let\PY@ff=\relax}
\def\PY@tok#1{\csname PY@tok@#1\endcsname}
\def\PY@toks#1+{\ifx\relax#1\empty\else%
    \PY@tok{#1}\expandafter\PY@toks\fi}
\def\PY@do#1{\PY@bc{\PY@tc{\PY@ul{%
    \PY@it{\PY@bf{\PY@ff{#1}}}}}}}
\def\PY#1#2{\PY@reset\PY@toks#1+\relax+\PY@do{#2}}

\@namedef{PY@tok@w}{\def\PY@tc##1{\textcolor[rgb]{0.73,0.73,0.73}{##1}}}
\@namedef{PY@tok@c}{\let\PY@it=\textit\def\PY@tc##1{\textcolor[rgb]{0.24,0.48,0.48}{##1}}}
\@namedef{PY@tok@cp}{\def\PY@tc##1{\textcolor[rgb]{0.61,0.40,0.00}{##1}}}
\@namedef{PY@tok@k}{\let\PY@bf=\textbf\def\PY@tc##1{\textcolor[rgb]{0.00,0.50,0.00}{##1}}}
\@namedef{PY@tok@kp}{\def\PY@tc##1{\textcolor[rgb]{0.00,0.50,0.00}{##1}}}
\@namedef{PY@tok@kt}{\def\PY@tc##1{\textcolor[rgb]{0.69,0.00,0.25}{##1}}}
\@namedef{PY@tok@o}{\def\PY@tc##1{\textcolor[rgb]{0.40,0.40,0.40}{##1}}}
\@namedef{PY@tok@ow}{\let\PY@bf=\textbf\def\PY@tc##1{\textcolor[rgb]{0.67,0.13,1.00}{##1}}}
\@namedef{PY@tok@nb}{\def\PY@tc##1{\textcolor[rgb]{0.00,0.50,0.00}{##1}}}
\@namedef{PY@tok@nf}{\def\PY@tc##1{\textcolor[rgb]{0.00,0.00,1.00}{##1}}}
\@namedef{PY@tok@nc}{\let\PY@bf=\textbf\def\PY@tc##1{\textcolor[rgb]{0.00,0.00,1.00}{##1}}}
\@namedef{PY@tok@nn}{\let\PY@bf=\textbf\def\PY@tc##1{\textcolor[rgb]{0.00,0.00,1.00}{##1}}}
\@namedef{PY@tok@ne}{\let\PY@bf=\textbf\def\PY@tc##1{\textcolor[rgb]{0.80,0.25,0.22}{##1}}}
\@namedef{PY@tok@nv}{\def\PY@tc##1{\textcolor[rgb]{0.10,0.09,0.49}{##1}}}
\@namedef{PY@tok@no}{\def\PY@tc##1{\textcolor[rgb]{0.53,0.00,0.00}{##1}}}
\@namedef{PY@tok@nl}{\def\PY@tc##1{\textcolor[rgb]{0.46,0.46,0.00}{##1}}}
\@namedef{PY@tok@ni}{\let\PY@bf=\textbf\def\PY@tc##1{\textcolor[rgb]{0.44,0.44,0.44}{##1}}}
\@namedef{PY@tok@na}{\def\PY@tc##1{\textcolor[rgb]{0.41,0.47,0.13}{##1}}}
\@namedef{PY@tok@nt}{\let\PY@bf=\textbf\def\PY@tc##1{\textcolor[rgb]{0.00,0.50,0.00}{##1}}}
\@namedef{PY@tok@nd}{\def\PY@tc##1{\textcolor[rgb]{0.67,0.13,1.00}{##1}}}
\@namedef{PY@tok@s}{\def\PY@tc##1{\textcolor[rgb]{0.73,0.13,0.13}{##1}}}
\@namedef{PY@tok@sd}{\let\PY@it=\textit\def\PY@tc##1{\textcolor[rgb]{0.73,0.13,0.13}{##1}}}
\@namedef{PY@tok@si}{\let\PY@bf=\textbf\def\PY@tc##1{\textcolor[rgb]{0.64,0.35,0.47}{##1}}}
\@namedef{PY@tok@se}{\let\PY@bf=\textbf\def\PY@tc##1{\textcolor[rgb]{0.67,0.36,0.12}{##1}}}
\@namedef{PY@tok@sr}{\def\PY@tc##1{\textcolor[rgb]{0.64,0.35,0.47}{##1}}}
\@namedef{PY@tok@ss}{\def\PY@tc##1{\textcolor[rgb]{0.10,0.09,0.49}{##1}}}
\@namedef{PY@tok@sx}{\def\PY@tc##1{\textcolor[rgb]{0.00,0.50,0.00}{##1}}}
\@namedef{PY@tok@m}{\def\PY@tc##1{\textcolor[rgb]{0.40,0.40,0.40}{##1}}}
\@namedef{PY@tok@gh}{\let\PY@bf=\textbf\def\PY@tc##1{\textcolor[rgb]{0.00,0.00,0.50}{##1}}}
\@namedef{PY@tok@gu}{\let\PY@bf=\textbf\def\PY@tc##1{\textcolor[rgb]{0.50,0.00,0.50}{##1}}}
\@namedef{PY@tok@gd}{\def\PY@tc##1{\textcolor[rgb]{0.63,0.00,0.00}{##1}}}
\@namedef{PY@tok@gi}{\def\PY@tc##1{\textcolor[rgb]{0.00,0.52,0.00}{##1}}}
\@namedef{PY@tok@gr}{\def\PY@tc##1{\textcolor[rgb]{0.89,0.00,0.00}{##1}}}
\@namedef{PY@tok@ge}{\let\PY@it=\textit}
\@namedef{PY@tok@gs}{\let\PY@bf=\textbf}
\@namedef{PY@tok@gp}{\let\PY@bf=\textbf\def\PY@tc##1{\textcolor[rgb]{0.00,0.00,0.50}{##1}}}
\@namedef{PY@tok@go}{\def\PY@tc##1{\textcolor[rgb]{0.44,0.44,0.44}{##1}}}
\@namedef{PY@tok@gt}{\def\PY@tc##1{\textcolor[rgb]{0.00,0.27,0.87}{##1}}}
\@namedef{PY@tok@err}{\def\PY@bc##1{{\setlength{\fboxsep}{\string -\fboxrule}\fcolorbox[rgb]{1.00,0.00,0.00}{1,1,1}{\strut ##1}}}}
\@namedef{PY@tok@kc}{\let\PY@bf=\textbf\def\PY@tc##1{\textcolor[rgb]{0.00,0.50,0.00}{##1}}}
\@namedef{PY@tok@kd}{\let\PY@bf=\textbf\def\PY@tc##1{\textcolor[rgb]{0.00,0.50,0.00}{##1}}}
\@namedef{PY@tok@kn}{\let\PY@bf=\textbf\def\PY@tc##1{\textcolor[rgb]{0.00,0.50,0.00}{##1}}}
\@namedef{PY@tok@kr}{\let\PY@bf=\textbf\def\PY@tc##1{\textcolor[rgb]{0.00,0.50,0.00}{##1}}}
\@namedef{PY@tok@bp}{\def\PY@tc##1{\textcolor[rgb]{0.00,0.50,0.00}{##1}}}
\@namedef{PY@tok@fm}{\def\PY@tc##1{\textcolor[rgb]{0.00,0.00,1.00}{##1}}}
\@namedef{PY@tok@vc}{\def\PY@tc##1{\textcolor[rgb]{0.10,0.09,0.49}{##1}}}
\@namedef{PY@tok@vg}{\def\PY@tc##1{\textcolor[rgb]{0.10,0.09,0.49}{##1}}}
\@namedef{PY@tok@vi}{\def\PY@tc##1{\textcolor[rgb]{0.10,0.09,0.49}{##1}}}
\@namedef{PY@tok@vm}{\def\PY@tc##1{\textcolor[rgb]{0.10,0.09,0.49}{##1}}}
\@namedef{PY@tok@sa}{\def\PY@tc##1{\textcolor[rgb]{0.73,0.13,0.13}{##1}}}
\@namedef{PY@tok@sb}{\def\PY@tc##1{\textcolor[rgb]{0.73,0.13,0.13}{##1}}}
\@namedef{PY@tok@sc}{\def\PY@tc##1{\textcolor[rgb]{0.73,0.13,0.13}{##1}}}
\@namedef{PY@tok@dl}{\def\PY@tc##1{\textcolor[rgb]{0.73,0.13,0.13}{##1}}}
\@namedef{PY@tok@s2}{\def\PY@tc##1{\textcolor[rgb]{0.73,0.13,0.13}{##1}}}
\@namedef{PY@tok@sh}{\def\PY@tc##1{\textcolor[rgb]{0.73,0.13,0.13}{##1}}}
\@namedef{PY@tok@s1}{\def\PY@tc##1{\textcolor[rgb]{0.73,0.13,0.13}{##1}}}
\@namedef{PY@tok@mb}{\def\PY@tc##1{\textcolor[rgb]{0.40,0.40,0.40}{##1}}}
\@namedef{PY@tok@mf}{\def\PY@tc##1{\textcolor[rgb]{0.40,0.40,0.40}{##1}}}
\@namedef{PY@tok@mh}{\def\PY@tc##1{\textcolor[rgb]{0.40,0.40,0.40}{##1}}}
\@namedef{PY@tok@mi}{\def\PY@tc##1{\textcolor[rgb]{0.40,0.40,0.40}{##1}}}
\@namedef{PY@tok@il}{\def\PY@tc##1{\textcolor[rgb]{0.40,0.40,0.40}{##1}}}
\@namedef{PY@tok@mo}{\def\PY@tc##1{\textcolor[rgb]{0.40,0.40,0.40}{##1}}}
\@namedef{PY@tok@ch}{\let\PY@it=\textit\def\PY@tc##1{\textcolor[rgb]{0.24,0.48,0.48}{##1}}}
\@namedef{PY@tok@cm}{\let\PY@it=\textit\def\PY@tc##1{\textcolor[rgb]{0.24,0.48,0.48}{##1}}}
\@namedef{PY@tok@cpf}{\let\PY@it=\textit\def\PY@tc##1{\textcolor[rgb]{0.24,0.48,0.48}{##1}}}
\@namedef{PY@tok@c1}{\let\PY@it=\textit\def\PY@tc##1{\textcolor[rgb]{0.24,0.48,0.48}{##1}}}
\@namedef{PY@tok@cs}{\let\PY@it=\textit\def\PY@tc##1{\textcolor[rgb]{0.24,0.48,0.48}{##1}}}

\def\PYZbs{\char`\\}
\def\PYZus{\char`\_}
\def\PYZob{\char`\{}
\def\PYZcb{\char`\}}
\def\PYZca{\char`\^}
\def\PYZam{\char`\&}
\def\PYZlt{\char`\<}
\def\PYZgt{\char`\>}
\def\PYZsh{\char`\#}
\def\PYZpc{\char`\%}
\def\PYZdl{\char`\$}
\def\PYZhy{\char`\-}
\def\PYZsq{\char`\'}
\def\PYZdq{\char`\"}
\def\PYZti{\char`\~}
% for compatibility with earlier versions
\def\PYZat{@}
\def\PYZlb{[}
\def\PYZrb{]}
\makeatother


    % For linebreaks inside Verbatim environment from package fancyvrb.
    \makeatletter
        \newbox\Wrappedcontinuationbox
        \newbox\Wrappedvisiblespacebox
        \newcommand*\Wrappedvisiblespace {\textcolor{red}{\textvisiblespace}}
        \newcommand*\Wrappedcontinuationsymbol {\textcolor{red}{\llap{\tiny$\m@th\hookrightarrow$}}}
        \newcommand*\Wrappedcontinuationindent {3ex }
        \newcommand*\Wrappedafterbreak {\kern\Wrappedcontinuationindent\copy\Wrappedcontinuationbox}
        % Take advantage of the already applied Pygments mark-up to insert
        % potential linebreaks for TeX processing.
        %        {, <, #, %, $, ' and ": go to next line.
        %        _, }, ^, &, >, - and ~: stay at end of broken line.
        % Use of \textquotesingle for straight quote.
        \newcommand*\Wrappedbreaksatspecials {%
            \def\PYGZus{\discretionary{\char`\_}{\Wrappedafterbreak}{\char`\_}}%
            \def\PYGZob{\discretionary{}{\Wrappedafterbreak\char`\{}{\char`\{}}%
            \def\PYGZcb{\discretionary{\char`\}}{\Wrappedafterbreak}{\char`\}}}%
            \def\PYGZca{\discretionary{\char`\^}{\Wrappedafterbreak}{\char`\^}}%
            \def\PYGZam{\discretionary{\char`\&}{\Wrappedafterbreak}{\char`\&}}%
            \def\PYGZlt{\discretionary{}{\Wrappedafterbreak\char`\<}{\char`\<}}%
            \def\PYGZgt{\discretionary{\char`\>}{\Wrappedafterbreak}{\char`\>}}%
            \def\PYGZsh{\discretionary{}{\Wrappedafterbreak\char`\#}{\char`\#}}%
            \def\PYGZpc{\discretionary{}{\Wrappedafterbreak\char`\%}{\char`\%}}%
            \def\PYGZdl{\discretionary{}{\Wrappedafterbreak\char`\$}{\char`\$}}%
            \def\PYGZhy{\discretionary{\char`\-}{\Wrappedafterbreak}{\char`\-}}%
            \def\PYGZsq{\discretionary{}{\Wrappedafterbreak\textquotesingle}{\textquotesingle}}%
            \def\PYGZdq{\discretionary{}{\Wrappedafterbreak\char`\"}{\char`\"}}%
            \def\PYGZti{\discretionary{\char`\~}{\Wrappedafterbreak}{\char`\~}}%
        }
        % Some characters . , ; ? ! / are not pygmentized.
        % This macro makes them "active" and they will insert potential linebreaks
        \newcommand*\Wrappedbreaksatpunct {%
            \lccode`\~`\.\lowercase{\def~}{\discretionary{\hbox{\char`\.}}{\Wrappedafterbreak}{\hbox{\char`\.}}}%
            \lccode`\~`\,\lowercase{\def~}{\discretionary{\hbox{\char`\,}}{\Wrappedafterbreak}{\hbox{\char`\,}}}%
            \lccode`\~`\;\lowercase{\def~}{\discretionary{\hbox{\char`\;}}{\Wrappedafterbreak}{\hbox{\char`\;}}}%
            \lccode`\~`\:\lowercase{\def~}{\discretionary{\hbox{\char`\:}}{\Wrappedafterbreak}{\hbox{\char`\:}}}%
            \lccode`\~`\?\lowercase{\def~}{\discretionary{\hbox{\char`\?}}{\Wrappedafterbreak}{\hbox{\char`\?}}}%
            \lccode`\~`\!\lowercase{\def~}{\discretionary{\hbox{\char`\!}}{\Wrappedafterbreak}{\hbox{\char`\!}}}%
            \lccode`\~`\/\lowercase{\def~}{\discretionary{\hbox{\char`\/}}{\Wrappedafterbreak}{\hbox{\char`\/}}}%
            \catcode`\.\active
            \catcode`\,\active
            \catcode`\;\active
            \catcode`\:\active
            \catcode`\?\active
            \catcode`\!\active
            \catcode`\/\active
            \lccode`\~`\~
        }
    \makeatother

    \let\OriginalVerbatim=\Verbatim
    \makeatletter
    \renewcommand{\Verbatim}[1][1]{%
        %\parskip\z@skip
        \sbox\Wrappedcontinuationbox {\Wrappedcontinuationsymbol}%
        \sbox\Wrappedvisiblespacebox {\FV@SetupFont\Wrappedvisiblespace}%
        \def\FancyVerbFormatLine ##1{\hsize\linewidth
            \vtop{\raggedright\hyphenpenalty\z@\exhyphenpenalty\z@
                \doublehyphendemerits\z@\finalhyphendemerits\z@
                \strut ##1\strut}%
        }%
        % If the linebreak is at a space, the latter will be displayed as visible
        % space at end of first line, and a continuation symbol starts next line.
        % Stretch/shrink are however usually zero for typewriter font.
        \def\FV@Space {%
            \nobreak\hskip\z@ plus\fontdimen3\font minus\fontdimen4\font
            \discretionary{\copy\Wrappedvisiblespacebox}{\Wrappedafterbreak}
            {\kern\fontdimen2\font}%
        }%

        % Allow breaks at special characters using \PYG... macros.
        \Wrappedbreaksatspecials
        % Breaks at punctuation characters . , ; ? ! and / need catcode=\active
        \OriginalVerbatim[#1,codes*=\Wrappedbreaksatpunct]%
    }
    \makeatother

    % Exact colors from NB
    \definecolor{incolor}{HTML}{303F9F}
    \definecolor{outcolor}{HTML}{D84315}
    \definecolor{cellborder}{HTML}{CFCFCF}
    \definecolor{cellbackground}{HTML}{F7F7F7}

    % prompt
    \makeatletter
    \newcommand{\boxspacing}{\kern\kvtcb@left@rule\kern\kvtcb@boxsep}
    \makeatother
    \newcommand{\prompt}[4]{
        {\ttfamily\llap{{\color{#2}[#3]:\hspace{3pt}#4}}\vspace{-\baselineskip}}
    }
    

    
    % Prevent overflowing lines due to hard-to-break entities
    \sloppy
    % Setup hyperref package
    \hypersetup{
      breaklinks=true,  % so long urls are correctly broken across lines
      colorlinks=true,
      urlcolor=urlcolor,
      linkcolor=linkcolor,
      citecolor=citecolor,
      }
    % Slightly bigger margins than the latex defaults
    
    \geometry{verbose,tmargin=1in,bmargin=1in,lmargin=1in,rmargin=1in}
    
    

\begin{document}
    
    \maketitle
    
    

    
    \[ \newcommand{\ket}[1]{\left|{#1}\right\rangle} \]
\[ \newcommand{\bra}[1]{\left\langle{#1}\right|} \]

    \section{Quantum Fourier Transform}\label{quantum-fourier-transform}

Linear, invertible transformation on qubits. The quantum analouge of
discrete Fourier transform. Requires only \(\mathcal{O}(nlog\ n)\) gates
to be implemented, and is a part of many important quantum algorithms
such as phase estimation.

\subsubsection{Motivation}\label{motivation}

A useful way to solve problems in many fields of science, especially in
physics and mathematics, is to transform it into some other (often
simpler) problem for which a solution is known. The discrete fourier
transform, which involves such a transformation, is one of a few known
algorithms that can be computed much faster on a quantum computer than
on a classical.

    \subsection{Fourier transform}\label{fourier-transform}

Assume a periodic function \(f(x)\) in an interval
\([ -\frac{L}{2}, \frac{L}{2} ]\). The fourier series in exponential
form can be written as \[
f(x) = \sum_{-\inf}^\inf A_n e^{i(2\pi nx/L)}
\] where \[
A_n = \frac{1}{L} \int_{-L/2}^{L/2} f(x)e^{-i(2\pi nx/L)} dx
\]

In the fourier transform \(A_n\) is transformed from a dicrete variable
to a continous one as \(L \rightarrow \inf\). We then replace \(A_n\)
with \(f(k)dk\) and let \(n/L \rightarrow k\), and the sum is changed to
an integral. This gives

\[
f(x) = \int_{-\inf}^{\inf}dkF(k) e^{i(2\pi kx)}
\]

\[
F(k) = \int_{-\inf}^{\inf}dxf(x) e^{-i(2\pi kx)}
\]

One way to interperet the Fourier transform is then as a transformation
from one basis to another.

    \subsection{Discrete Fourier
transform}\label{discrete-fourier-transform}

Next we make another generalization by having a discrete function, that
is \(f(x) \rightarrow f(x_k)\) with \(x_k = k\Delta x\) for
\(k=0, \dots, N-1\). This leads to the sums

\[
f_x = \frac{1}{N} \sum_{k=0}^{N-1}F_k e^{i(2\pi kx)/N}
\]

\[
F_k = \sum_{x=0}^{N-1}f_x e^{-i(2\pi kx)/N}
\]

Although we have used functions here, this could also be a set of
numbers. As an example we can have a set of complex numbers
\(\{ x_0,\dots,x_{N-1}\}\) with fixed length \(N\), we can Fourier
transform this as \begin{equation}
y_k = \frac{1}{\sqrt{N}} \sum_{j=0}^{N-1} x_j e^{i(2\pi jk)/N}
\end{equation} leading to a new set of complex numbers
\(\{ y_0,\dots,y_{N-1}\}\).

    \subsection{Quantum Fourier transform}\label{quantum-fourier-transform}

We now turn to the quantum Fourier transform. It is the same
transformation as described above, however we define it in terms of the
unitary operation

\[
    \ket{\psi'} \leftarrow \hat{F}\ket{\psi}, \quad \hat{F}^\dagger \hat{F} = I
\]

In terms of an orthonormal basis \(\ket{0},\ket{1},\dots,\ket{0}\) this
linear operator has the following action

\[
\ket{j} \rightarrow \sum_{k=0}^{N-1} e^{i(2\pi jk/N)}\ket{k}
\]

or on an arbitrary state

\$\$ \sum\emph{\{j=0\}\^{}\{N-1\} x\_j \ket{j}
\rightarrow \sum}\{k=0\}\^{}\{N-1\} y\_k\ket{k}

\$\$

equivalent to the equation for discrete Fourier transform on a set of
complex numbers.

Next we assume an \(n\)-qubit system, where we take \(N=s^n\) in the
computational basis \$ \ket{0},\dots,\ket{2^n -1} \$ We make use of the
binary representation
\(j = j_1 2^{n-1} + j_2 2^{n-2} + \dots + j_n 2^0\) , and take note of
the notation \(0.j_l j_{l+1} \dots j_m\)~representing the binary
fraction
\(\frac{j_l}{2^1} + \frac{j_{l+1}}{2^{2}} + \dots + \frac{j_m}{2^{m-l+1}}\).
With this we define the product representation of the quantum Fourier
transform

\[
\ket{j_1,\dots,j_n} \rightarrow 
\frac{
\left(\ket{0} + e^{i(2\pi 0.j_n)}\right)
\left(\ket{0} + e^{i(2\pi 0.j_{j-1}j_n)}\right)
\dots
\left(\ket{0} + e^{i(2\pi 0.j_1j_2\dots j_n)}\right)
}{2^{n/2}}
\]

\subsubsection{Components}\label{components}

From the product representation we can derive a circuit for the quantum
Fourier transform. This will make use of the following two single-qubit
gates

\begin{equation}
    H = \frac{1}{\sqrt{2}}
    \begin{bmatrix}
        1 & 1 \\
        1 & -1
    \end{bmatrix}
\end{equation}

\begin{equation}
    R_k =
    \begin{bmatrix}
        1 & 0 \\
        0 & e^{2\pi i/2^{k}}
    \end{bmatrix}
\end{equation}

First we refresh our memory of the action of these gates. The hadamard
gate on a single qubit creates an equal superposition of its basis
states, assuming it is not already in a superposition, such that

\[
    H\ket{0} = \frac{1}{\sqrt{2}} \left(\ket{0} + \ket{1}\right), \quad H\ket{1} = \frac{1}{\sqrt{2}} \left(\ket{0} - \ket{1}\right)
\]

The \(R_k\)~gate simply adds a phase if the qubit it acts on is in the
state \(\ket{1}\)

\[
    R_k\ket{0} = \ket{0}, \quad R_k\ket{1} = e^{2\pi i/2^{k}}\ket{1}
\]

Since all this gates are unitary, the quantum Fourier transfrom is also
unitary.

\subsubsection{Algorithm}\label{algorithm}

Assume we have a quantum register of \(n\) qubits in the state
\(\ket{j_1 j_2 \dots j_n}\). Applying the hadamard gate to the first
qubit produces the state

\[
\\
H\ket{j_1 j_2 \dots j_n} = \frac{\left(\ket{0} + e^{2\pi i 0.j_1}\ket{1}\right)}{2^{1/2}} \ket{j_2 \dots j_n}
\\
\]

where we have made use of the binary fraction to represent the action of
the hadamard gate

\[
\begin{equation}
e^{2\pi i 0.j_1} = 
\begin{cases}
-1, & \quad \text{if $j_1 = 1$} \\
+1, & \quad \text{if $j_1 = 0$}
\end{cases}
\end{equation}
\]

Furthermore we can apply the controlled-\(R_k\)~gate, with all the other
qubits \(j_k\) for \(k>1\) as control qubits to produce the state

\[
\\
\frac{\left(\ket{0} + e^{2\pi i 0.j_1j_2\dots j_n}\ket{1}\right)}{2^{1/2}} \ket{j_2 \dots j_n}
\\
\]

Next we do the same procedure on qubit \(2\)~producing the state

\[
\\
\frac{\left(\ket{0} + e^{2\pi i 0.j_1j_2\dots j_n}\ket{1}\right)\left(\ket{0} + e^{2\pi i 0.j_2\dots j_n}\ket{1}\right)}{2^{2/2}} \ket{j_2 \dots j_n}
\\
\]

Doing this for all \(n\) qubits yields state

\[
\\
\frac{\left(\ket{0} + e^{2\pi i 0.j_1j_2\dots j_n}\ket{1}\right)\left(\ket{0} + e^{2\pi i 0.j_2\dots j_n}\ket{1}\right)\dots \left(\ket{0} + e^{2\pi i 0.j_n}\ket{1}\right)}{2^{n/2}} \ket{j_2 \dots j_n}
\\
\]

At the end we use swap gates to reverse the order of the qubits

\[
\\
\frac{\left(\ket{0} + e^{2\pi i 0.j_n}\ket{1}\right)\left(\ket{0} + e^{2\pi i 0.j_{n-1}j_n}\ket{1}\right)\dots\left(\ket{0} + e^{2\pi i 0.j_1j_2\dots j_n}\ket{1}\right) }{2^{n/2}} \ket{j_2 \dots j_n}
\\
\]

This is just the product representation from earlier, obviously our
desired output.

    \begin{tcolorbox}[breakable, size=fbox, boxrule=1pt, pad at break*=1mm,colback=cellbackground, colframe=cellborder]
\prompt{In}{incolor}{5}{\boxspacing}
\begin{Verbatim}[commandchars=\\\{\}]
\PY{k+kn}{import} \PY{n+nn}{qiskit} \PY{k}{as} \PY{n+nn}{qk}
\PY{k+kn}{import} \PY{n+nn}{numpy} \PY{k}{as} \PY{n+nn}{np}
\PY{c+c1}{\PYZsh{}qk.IBMQ.load\PYZus{}account()}

\PY{k}{def} \PY{n+nf}{QFT}\PY{p}{(}\PY{n}{Qcircuit}\PY{p}{,} \PY{n}{inverse}\PY{o}{=}\PY{k+kc}{False}\PY{p}{)}\PY{p}{:}
\PY{+w}{    }\PY{l+s+sd}{\PYZdq{}\PYZdq{}\PYZdq{} \PYZus{}\PYZus{}\PYZus{}\PYZus{}\PYZus{}\PYZus{}\PYZus{}\PYZus{}\PYZus{}\PYZus{}\PYZus{}\PYZus{}\PYZus{}\PYZus{}\PYZus{}\PYZus{}\PYZus{}\PYZus{}\PYZus{}\PYZus{}\PYZus{}\PYZus{}\PYZus{}\PYZus{}\PYZus{}}
\PY{l+s+sd}{    }
\PY{l+s+sd}{        Quantum Fourier Transform}
\PY{l+s+sd}{        \PYZus{}\PYZus{}\PYZus{}\PYZus{}\PYZus{}\PYZus{}\PYZus{}\PYZus{}\PYZus{}\PYZus{}\PYZus{}\PYZus{}\PYZus{}\PYZus{}\PYZus{}\PYZus{}\PYZus{}\PYZus{}\PYZus{}\PYZus{}\PYZus{}\PYZus{}\PYZus{}\PYZus{}\PYZus{}}
\PY{l+s+sd}{        }
\PY{l+s+sd}{        Input: }
\PY{l+s+sd}{        }
\PY{l+s+sd}{            Qcircuit = [qc,qr,cr,n]}
\PY{l+s+sd}{                \PYZhy{} qc \PYZhy{}\PYZgt{} Quantum circuit object}
\PY{l+s+sd}{                \PYZhy{} qr \PYZhy{}\PYZgt{} Quantum register object}
\PY{l+s+sd}{                \PYZhy{} cr \PYZhy{}\PYZgt{} Classical register object}
\PY{l+s+sd}{                \PYZhy{} n  \PYZhy{}\PYZgt{} Number of qubits}
\PY{l+s+sd}{                }
\PY{l+s+sd}{            inverse:}
\PY{l+s+sd}{                True,False}
\PY{l+s+sd}{   }
\PY{l+s+sd}{        Output:}
\PY{l+s+sd}{        }
\PY{l+s+sd}{            Qcircuit}
\PY{l+s+sd}{    \PYZdq{}\PYZdq{}\PYZdq{}}
    
    \PY{n}{qc}       \PY{o}{=}  \PY{n}{Qcircuit}\PY{p}{[}\PY{l+m+mi}{0}\PY{p}{]}
    \PY{n}{qr}       \PY{o}{=}  \PY{n}{Qcircuit}\PY{p}{[}\PY{l+m+mi}{1}\PY{p}{]}
    \PY{n}{n\PYZus{}qubits} \PY{o}{=}  \PY{n}{Qcircuit}\PY{p}{[}\PY{l+m+mi}{2}\PY{p}{]}
    
    \PY{k}{if} \PY{o+ow}{not} \PY{n}{inverse}\PY{p}{:}
        \PY{k}{for} \PY{n}{i} \PY{o+ow}{in} \PY{n+nb}{range}\PY{p}{(}\PY{n}{n\PYZus{}qubits}\PY{p}{)}\PY{p}{:}
            \PY{n}{qc}\PY{o}{.}\PY{n}{h}\PY{p}{(}\PY{n}{qr}\PY{p}{[}\PY{n}{i}\PY{p}{]}\PY{p}{)}
            \PY{k}{for} \PY{n}{j} \PY{o+ow}{in} \PY{n+nb}{range}\PY{p}{(}\PY{n}{i}\PY{o}{+}\PY{l+m+mi}{1}\PY{p}{,}\PY{n}{n\PYZus{}qubits}\PY{p}{)}\PY{p}{:}
                \PY{n}{qc}\PY{o}{.}\PY{n}{cu1}\PY{p}{(}\PY{n}{np}\PY{o}{.}\PY{n}{pi}\PY{o}{/}\PY{l+m+mi}{2}\PY{o}{*}\PY{o}{*}\PY{p}{(}\PY{n}{j}\PY{o}{\PYZhy{}}\PY{n}{i}\PY{p}{)}\PY{p}{,}\PY{n}{qr}\PY{p}{[}\PY{n}{j}\PY{p}{]}\PY{p}{,}\PY{n}{qr}\PY{p}{[}\PY{n}{i}\PY{p}{]}\PY{p}{)}

        \PY{k}{for} \PY{n}{i} \PY{o+ow}{in} \PY{n+nb}{range}\PY{p}{(}\PY{n+nb}{int}\PY{p}{(}\PY{n}{n\PYZus{}qubits}\PY{o}{/}\PY{l+m+mi}{2}\PY{p}{)}\PY{p}{)}\PY{p}{:}
            \PY{n}{qc}\PY{o}{.}\PY{n}{swap}\PY{p}{(}\PY{n}{qr}\PY{p}{[}\PY{n}{i}\PY{p}{]}\PY{p}{,}\PY{n}{qr}\PY{p}{[}\PY{o}{\PYZhy{}}\PY{p}{(}\PY{n}{i}\PY{o}{+}\PY{l+m+mi}{1}\PY{p}{)}\PY{p}{]}\PY{p}{)}
    \PY{k}{else}\PY{p}{:}
        \PY{k}{for} \PY{n}{i} \PY{o+ow}{in} \PY{n+nb}{range}\PY{p}{(}\PY{n+nb}{int}\PY{p}{(}\PY{n}{n\PYZus{}qubits}\PY{o}{/}\PY{l+m+mi}{2}\PY{p}{)}\PY{p}{)}\PY{p}{:}
            \PY{n}{qc}\PY{o}{.}\PY{n}{swap}\PY{p}{(}\PY{n}{qr}\PY{p}{[}\PY{n}{i}\PY{p}{]}\PY{p}{,}\PY{n}{qr}\PY{p}{[}\PY{o}{\PYZhy{}}\PY{p}{(}\PY{n}{i}\PY{o}{+}\PY{l+m+mi}{1}\PY{p}{)}\PY{p}{]}\PY{p}{)}
            
        \PY{k}{for} \PY{n}{i} \PY{o+ow}{in} \PY{n+nb}{range}\PY{p}{(}\PY{n}{n\PYZus{}qubits}\PY{p}{)}\PY{p}{:}
            \PY{k}{for} \PY{n}{j} \PY{o+ow}{in} \PY{n+nb}{range}\PY{p}{(}\PY{n}{i}\PY{p}{)}\PY{p}{:}
                \PY{n}{qc}\PY{o}{.}\PY{n}{cu1}\PY{p}{(}\PY{o}{\PYZhy{}}\PY{n}{np}\PY{o}{.}\PY{n}{pi}\PY{o}{/}\PY{l+m+mi}{2}\PY{o}{*}\PY{o}{*}\PY{p}{(}\PY{n}{i}\PY{o}{\PYZhy{}}\PY{n}{j}\PY{p}{)}\PY{p}{,}\PY{n}{qr}\PY{p}{[}\PY{n}{j}\PY{p}{]}\PY{p}{,}\PY{n}{qr}\PY{p}{[}\PY{n}{i}\PY{p}{]}\PY{p}{)}
            \PY{n}{qc}\PY{o}{.}\PY{n}{h}\PY{p}{(}\PY{n}{qr}\PY{p}{[}\PY{n}{i}\PY{p}{]}\PY{p}{)}    
    
    \PY{k}{return} \PY{p}{[}\PY{n}{qc}\PY{p}{,}\PY{n}{qc}\PY{p}{,}\PY{n}{n\PYZus{}qubits}\PY{p}{]} 
\end{Verbatim}
\end{tcolorbox}

    \begin{tcolorbox}[breakable, size=fbox, boxrule=1pt, pad at break*=1mm,colback=cellbackground, colframe=cellborder]
\prompt{In}{incolor}{2}{\boxspacing}
\begin{Verbatim}[commandchars=\\\{\}]
\PY{c+c1}{\PYZsh{}\PYZsh{}\PYZsh{} Simple 3\PYZhy{}qubit transform to confirm correct implementation}
\PY{n}{n\PYZus{}qubits} \PY{o}{=} \PY{l+m+mi}{3}
\PY{n}{qr1}      \PY{o}{=} \PY{n}{qk}\PY{o}{.}\PY{n}{QuantumRegister}\PY{p}{(}\PY{n}{n\PYZus{}qubits}\PY{p}{)}
\PY{n}{qc1}      \PY{o}{=} \PY{n}{qk}\PY{o}{.}\PY{n}{QuantumCircuit}\PY{p}{(}\PY{n}{qr1}\PY{p}{)}
\PY{n}{qr2}      \PY{o}{=} \PY{n}{qk}\PY{o}{.}\PY{n}{QuantumRegister}\PY{p}{(}\PY{n}{n\PYZus{}qubits}\PY{p}{)}
\PY{n}{qc2}      \PY{o}{=} \PY{n}{qk}\PY{o}{.}\PY{n}{QuantumCircuit}\PY{p}{(}\PY{n}{qr2}\PY{p}{)}
\PY{n}{Qcircuit1} \PY{o}{=} \PY{n}{QFT}\PY{p}{(}\PY{p}{[}\PY{n}{qc1}\PY{p}{,}\PY{n}{qr1}\PY{p}{,}\PY{n}{n\PYZus{}qubits}\PY{p}{]}\PY{p}{)}
\PY{n}{Qcircuit2} \PY{o}{=} \PY{n}{QFT}\PY{p}{(}\PY{p}{[}\PY{n}{qc2}\PY{p}{,}\PY{n}{qr2}\PY{p}{,}\PY{n}{n\PYZus{}qubits}\PY{p}{]}\PY{p}{,}\PY{n}{inverse}\PY{o}{=}\PY{k+kc}{True}\PY{p}{)}
\end{Verbatim}
\end{tcolorbox}

    \begin{Verbatim}[commandchars=\\\{\}, frame=single, framerule=2mm, rulecolor=\color{outerrorbackground}]
\textcolor{ansi-red}{---------------------------------------------------------------------------}
\textcolor{ansi-red}{NameError}                                 Traceback (most recent call last)
Input \textcolor{ansi-green}{In [2]}, in \textcolor{ansi-cyan}{<cell line: 7>}\textcolor{ansi-blue}{()}
\textcolor{ansi-green-intense}{\textbf{      5}} qr2      \def\tcRGB{\textcolor[RGB]}\expandafter\tcRGB\expandafter{\detokenize{98,98,98}}{=} qk\def\tcRGB{\textcolor[RGB]}\expandafter\tcRGB\expandafter{\detokenize{98,98,98}}{.}QuantumRegister(n\_qubits)
\textcolor{ansi-green-intense}{\textbf{      6}} qc2      \def\tcRGB{\textcolor[RGB]}\expandafter\tcRGB\expandafter{\detokenize{98,98,98}}{=} qk\def\tcRGB{\textcolor[RGB]}\expandafter\tcRGB\expandafter{\detokenize{98,98,98}}{.}QuantumCircuit(qr2)
\textcolor{ansi-green}{----> 7} Qcircuit1 \def\tcRGB{\textcolor[RGB]}\expandafter\tcRGB\expandafter{\detokenize{98,98,98}}{=} \setlength{\fboxsep}{0pt}\colorbox{ansi-yellow}{QFT\strut}([qc1,qr1,n\_qubits])
\textcolor{ansi-green-intense}{\textbf{      8}} Qcircuit2 \def\tcRGB{\textcolor[RGB]}\expandafter\tcRGB\expandafter{\detokenize{98,98,98}}{=} QFT([qc2,qr2,n\_qubits],inverse\def\tcRGB{\textcolor[RGB]}\expandafter\tcRGB\expandafter{\detokenize{98,98,98}}{=}\def\tcRGB{\textcolor[RGB]}\expandafter\tcRGB\expandafter{\detokenize{0,135,0}}{\textbf{True}})

\textcolor{ansi-red}{NameError}: name 'QFT' is not defined
    \end{Verbatim}

    \begin{tcolorbox}[breakable, size=fbox, boxrule=1pt, pad at break*=1mm,colback=cellbackground, colframe=cellborder]
\prompt{In}{incolor}{3}{\boxspacing}
\begin{Verbatim}[commandchars=\\\{\}]
\PY{n}{Qcircuit1}\PY{p}{[}\PY{l+m+mi}{0}\PY{p}{]}\PY{o}{.}\PY{n}{draw}\PY{p}{(}\PY{p}{)}
\end{Verbatim}
\end{tcolorbox}

    \begin{Verbatim}[commandchars=\\\{\}, frame=single, framerule=2mm, rulecolor=\color{outerrorbackground}]
\textcolor{ansi-red}{---------------------------------------------------------------------------}
\textcolor{ansi-red}{NameError}                                 Traceback (most recent call last)
Input \textcolor{ansi-green}{In [3]}, in \textcolor{ansi-cyan}{<cell line: 1>}\textcolor{ansi-blue}{()}
\textcolor{ansi-green}{----> 1} \setlength{\fboxsep}{0pt}\colorbox{ansi-yellow}{Qcircuit1\strut}[\def\tcRGB{\textcolor[RGB]}\expandafter\tcRGB\expandafter{\detokenize{98,98,98}}{0}]\def\tcRGB{\textcolor[RGB]}\expandafter\tcRGB\expandafter{\detokenize{98,98,98}}{.}draw()

\textcolor{ansi-red}{NameError}: name 'Qcircuit1' is not defined
    \end{Verbatim}

    \begin{tcolorbox}[breakable, size=fbox, boxrule=1pt, pad at break*=1mm,colback=cellbackground, colframe=cellborder]
\prompt{In}{incolor}{30}{\boxspacing}
\begin{Verbatim}[commandchars=\\\{\}]
\PY{n}{Qcircuit2}\PY{p}{[}\PY{l+m+mi}{0}\PY{p}{]}\PY{o}{.}\PY{n}{draw}\PY{p}{(}\PY{p}{)}
\end{Verbatim}
\end{tcolorbox}

            \begin{tcolorbox}[breakable, size=fbox, boxrule=.5pt, pad at break*=1mm, opacityfill=0]
\prompt{Out}{outcolor}{30}{\boxspacing}
\begin{Verbatim}[commandchars=\\\{\}]
<qiskit.visualization.text.TextDrawing at 0x7f75c7b27c18>
\end{Verbatim}
\end{tcolorbox}
        
    \begin{tcolorbox}[breakable, size=fbox, boxrule=1pt, pad at break*=1mm,colback=cellbackground, colframe=cellborder]
\prompt{In}{incolor}{ }{\boxspacing}
\begin{Verbatim}[commandchars=\\\{\}]

\end{Verbatim}
\end{tcolorbox}

    \begin{tcolorbox}[breakable, size=fbox, boxrule=1pt, pad at break*=1mm,colback=cellbackground, colframe=cellborder]
\prompt{In}{incolor}{ }{\boxspacing}
\begin{Verbatim}[commandchars=\\\{\}]

\end{Verbatim}
\end{tcolorbox}


    % Add a bibliography block to the postdoc
    
    
    
\end{document}
